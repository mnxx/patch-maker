\documentclass[a4paper, 10pt, french]{article}
% Préambule; packages qui peuvent être utiles
   \RequirePackage[T1]{fontenc}        % Ce package pourrit les pdf...
   \RequirePackage{babel,indentfirst}  % Pour les césures correctes,
                                       % et pour indenter au début de chaque paragraphe
   \RequirePackage[utf8]{inputenc}   % Pour pouvoir utiliser directement les accents
                                     % et autres caractères français
   \RequirePackage{lmodern,tgpagella} % Police de caractères
   \textwidth 17cm \textheight 25cm \oddsidemargin -0.24cm % Définition taille de la page
   \evensidemargin -1.24cm \topskip 0cm \headheight -1.5cm % Définition des marges
   \RequirePackage{latexsym}                  % Symboles
   \RequirePackage{amsmath}                   % Symboles mathématiques
   \RequirePackage{tikz}   % Pour faire des schémas
   \RequirePackage{graphicx} % Pour inclure des images
   \RequirePackage{listings} % pour mettre des listings
% Fin Préambule; package qui peuvent être utiles

\title{Rapport de TP 4MMAOD : Génération de patch optimal}
\author{
MAILLOT Franck ISSC (groupe 105) 
\\ ROTH Manuel ISSC (groupe 105) 
}

\begin{document}

\maketitle

%%%%%%%%%%%%%%%%%%%%%%%%%%%%%%%%%%%%%%%%%%%%%%
%\paragraph{\em Préambule}
%{\em \begin{itemize} 
%   \item Compléter ce patron de rapport en supprimant toutes les phrases en italique\,: elles ne doivent pas apparaître dans le rapport pdf.
%   \item Il sera attribué {\bf 1 point} pour la qualité globale du rapport\,: présentation, concision et clarté de l'argumentation.
%\end{itemize}
%}

%%%%%%%%%%%%%%%%%%%%%%%%%%%%%%%%%%%%%%%%%%%%%%
\section{Principe de notre  programme (1 point)}
%{\em Mettre ici une explication brève du principe de votre programme en  précisant la méthode implantée (récursive, itérative) et les
%choix effectués (notamment pour l'ordonnancement des instructions).
%}
Notre programme résout la problématique de créer un patch optimal avec une approche itérative. Pour cela, on utilise un tableau bidimensionnel,
qui nous permet de rapidement trouver la sous-solution optimale. L'affichage des commandes à effectuer est créée de fa\c{c}on récursive.

%%%%%%%%%%%%%%%%%%%%%%%%%%%%%%%%%%%%%%%%%%%%%%
\section{Analyse du coût théorique (3 points)}
%{\em Donner ici l'analyse du coût théorique de votre programme en fonction des nombres $n_1$ et $n_2$ de lignes 
%et $c_1$ et $c_2$ de caractères des deux fichiers en entrée.
% Pour chaque coût, donner la formule qui le caractérise (en justifiant brièvement pourquoi cette formule correspond à votre programme), 
% puis l'ordre du coût en fonction de $n_1, n_2, c_1, c_2$ en notation $\Theta$ de préférence, sinon $O$.}
L'enregistrement des .. est effectué par un tableau bidimensionnel. Ce tableau est initialisé par une bouce for

\subsection{Nombre  d'opérations en pire cas\,: }
Avec notre implémentation, on a un coût de:\\
\[
\Theta(n_1,n_2,c_1,c_2) = n_1 \cdot n_2 + n_1 + n_2 
\]
    \paragraph{Justification\,: }
    {\em La justification peut être par exemple de la forme: \\ 
       "Le programme itératif contient les boucles $k_1=...$, $k_2= ...$ etc correspondant à la somme 
      $$T(n_1, n_2, c_1, c_2) = \sum_{k_1=...}^{...} ... \sum ... + \sum_{i=...}^{...} ...$$ 
      somme que nous avons calculée (ou majorée) par la technique de  ... " \\
      ou  encore\,:  \\
      "les appels récursifs du programme permettent de modéliser son coût par le système d'équations aux récurrences 
      $$T(k_1, k_2) = ...  \mbox{~avec~les~conditions~initiales~....~} $$
      Le coût indiqué est obtenu en résolvant ce système par la méthode de  .... "
    } 
  \subsection{Place mémoire requise\,: }
    \paragraph{Justification\,: }

  \subsection{Nombre de défauts de cache sur le modèle CO\,: }
    \paragraph{Justification\,: }


%%%%%%%%%%%%%%%%%%%%%%%%%%%%%%%%%%%%%%%%%%%%%%
\section{Compte rendu d'expérimentation (2 points)}
  \subsection{Conditions expérimentaless}
     %{\em Décrire les conditions permettant la reproductibilité des mesures: on demande la description
     % de la machine et la méthode utilisée pour mesurer le temps.
     %}

    \subsubsection{Description synthétique de la machine\,:} 
      %{\em indiquer ici le  processeur et sa fréquence, la mémoire, le système d'exploitation. 
       %Préciser aussi si la machine était monopolisée pour un test, ou notamment si 
       %d'autres processus ou utilisateurs étaient en cours d'exécution. 
      %} 
Le processeur utilisé est un Intel Core i5-4690 CPU @ 3,9 GHz.\\
Ce processeur a une fréquence de base de 3,5 GHz et un Smart Cache de 6 MB.\\
L'ordinateur a une mémoire RAM de 15992 MiB, une mémoire SSD de 250 GiB et une mémoire HDD de 1 TiB.\\
Le système d'exploitation est Linux Mint 17.2 Rafaela.\\
On n'avait pas d'autres processus en exécution.

    \subsubsection{Méthode utilisée pour les mesures de temps\,: } 
      %{\em préciser ici  comment les mesures de temps ont été effectuées (fonction appelée) et l'unité de temps; en particulier, 
      % préciser comment les 5 exécutions pour chaque test ont été faites (par exemple si le même test est fait 5 fois de suite, ou si les tests sont alternés entre
      % les mesures, ou exécutés en concurrence etc). 
      %}
    Les tests ont été effectuées avec la fontion :
\\
\$ time bin/computePatchOpt benchmark/benchmark\_0X/source benchmark/benchmark\_0X/target 
\\
Ce commande nous a retourné le temps d'exécution en secondes avec une précision d'une milliseconde.\\
Pour chacun des benchmarks, on a effectué une série de cinq tests. Pour une meilleure visibilité, les valeurs dans le tableau suivant sont en millisecondes.

  \subsection{Mesures expérimentales}
    Le tableau pour les temps d'exécution mesurés pour chacun des 6 benchmarks imposés (temps minimum, maximum et moyen sur 5 exécutions) :

    \begin{figure}[h]
      \begin{center}
        \begin{tabular}{|l||r||r|r|r||}
          \hline
          \hline
            & coût         & temps     & temps   & temps \\
            & du patch     & min       & max     & moyen \\
          \hline
          \hline
            benchmark1 & $1000\cdot1047 \approx 10^6$     & 31ms    & 86ms    & 56ms    \\
          \hline
            benchmark2 & $2047\cdot2094 \approx 4\cdot10^6$    & 107ms    & 238ms    & 165ms    \\
          \hline
            benchmark3 & $3000\cdot3004 \approx 9\cdot10^6$    & 281ms    & 363ms    & 324ms    \\
          \hline
            benchmark4 & $4000\cdot4010 \approx 1.6\cdot10^7$    & 368ms    & 492ms    & 410ms    \\
          \hline
            benchmark5 & $4999\cdot4986 \approx 2.5\cdot10^7$    & 623ms    & 715ms    & 658ms     \\
          \hline
            benchmark6 & $10000\cdot10694 \approx 10^8$     & 2646ms    & 2746ms    & 2683ms    \\
          \hline
          \hline
        \end{tabular}
        \caption{ Mesures des temps minimum, maximum et moyen de 5 exécutions pour les 6 benchmarks.}
        \label{table-temps}
      \end{center}
    \end{figure}

\subsection{Analyse des résultats expérimentaux}
%{\em Donner  une réponse justifiée  à la question\,: 
%              les  temps mesurés correspondent ils  à votre analyse théorique (nombre d’opérations et défauts de cache) ?
%}
En général, nos résultats correspondent à nos calculs théoriques. Si on compare le nombre de lignes des fichiers avec le temps de calcul,
on voit bien qu'il s'agit d'une correspondance linéaire.
\\
\\
Par exemple, pour le benchmark\_06 qui a environ 2 fois plus de lignes par fichier que le benchmark\_05 :\\ 
$n_{06} \cdot m_{06} = 2 \cdot n_{05} \cdot 2 \cdot m_{05} = 4 \cdot n_{05} \cdot m_{05}$.\\
Les temps de calcul valent respectivement :\\ 
$t_{06} = 2683 ms \quad$ est très proche de $\quad 2632 ms = 4 \cdot t_{05}$.\\
\\
Bien sûr, les résultats ne sont pas toutes si exactes, à cause des variations des tests et le fait, qu'on a pris la moyenne de que cinq tests.
Néanmoins, nous pensons que nos résultats confirment bien nos calculs théoriques.
\\

%%%%%%%%%%%%%%%%%%%%%%%%%%%%%%%%%%%%%%%%%%%%%%
\section{Question\,: et  si le coût d'un patch était sa taille en octets ? (1 point)}
%{\em Préciser le principe de la résolution choisie (parmi celles vues en cours); donner  les modifications à apporter (soit à vos  équations, soit à votre programme, au choix) 
%pour s'adapter à cette nouvelle fonction de coût. 
On utilise exactement la même méthode que précédemment en changeant les formules de coût.\\
\\
Ce qui joue maintenant un rôle est le nombre de caractères pour chaque ligne du patch :\\
\indent - Seul le coût d'une copie reste inchangé.\\
\indent - La formule pour le coût de l'addition de la ligne j après la ligne i est :\\
\indent  $3 + (log_{10}(i) + 1) + strlen(targetLine[j]);$\\
\indent - La formule pour le coût de la substitution de la ligne i par la ligne j est de même :\\
\indent  $3 + (log_{10}(i) + 1) + strlen(targetLine[j]);$\\
\indent - La coût de la destruction de la ligne i devient :\\
\indent  $3 + (log(i) + 1);$\\
\indent - La d'une destruction multiple de k ligne(s) à partir de la ligne i devient :\\
\indent  $4 + (log(i) + 1) + (log(k) + 1);$\\
\\
Avec :\\
  - $3$ correspond à la taille d'un $'+'$, $'='$ ou $'d'$, d'un espace $'\;'$ et d'un $'\textbackslash n'$ en fin de ligne.\\
  - $4$ correspond à la taille d'un $'D'$,de deux espaces $'\quad'$ et d'un $'\textbackslash n'$ en fin de ligne.\\
  - $(log_{10}(i) + 1)$ correspond au nombre de caractères nécessaires pour représenter le nombre i.\\
\\
La majorité de ces formules peuvent remplacés celles déjà en place dans notre code, à l'exception
des destructions multiples. En effet le fonctionnement actuel de notre code ne considère que le
plus petit coût parmis les i-2 cases précédentes lors du calcul pour une case (j, i) donnée, en 
supposant que le coût à ajouter est 15, donc le même pour tous, ce qui ne serait plus le cas car 
le coût pour passer de la case (j, k) à la case (j, i) devient fonction de k et (i-k).
\\
}

\end{document}
%% Fin mise au format

